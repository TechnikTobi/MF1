\documentclass[../../main.tex]{subfiles}

\begin{document}
	
	\chapter{Relationen}
	
	\section{Grundlegendes}
	
	\begin{definition}[(2-stellige) Relation]
		\label{def:Relation}
		Seien $M$ und $N$ \hyperref[def:Menge]{Mengen} und $R$ eine \hyperref[def:Teilmenge]{Teilmenge} des kartesischen Produkt der beiden ($R \subseteq M \times N$). Nun verstehen wir unter $R$ eine \textbf{Relation} auf $M \times N$. Für den Spezialfall $M=N$ heißt $R$ Relation auf $M$.\sidenote{Der Punkt dabei ist, dass die \hyperref[def:Teilmenge]{Teilmenge} $R$ beliebig definiert werden kann um verschiedenste Relationen bilden zu können.}
	\end{definition}

	Dabei gilt es vor allem zu beachten, dass eine Relation für ein Paar von Werten (= ein Element des kartesischen Produkts) nur entweder zutreffen kann oder nicht - entweder es gibt eine Relation zwischen den Werten oder nicht. Wir können eine Relation also als eine Aussagevorschrift über das Verhältnis zwischen diesen Werten betrachten.\\
	
	
	Um auszudrücken dass beispielsweise die Elemente $0$ und $1$ die Relation $<$ erfüllen können wir eine der folgenden Schreibweisen verwenden: $R_<(0,1)$\sidenote{Es kann auch nur $R(0,1)$ geschrieben werden, vorausgesetzt die Bezeichnung $R$ ist eindeutig} oder $(0,1) \in R_<$ oder schlicht $0<1$.
	
	\begin{definition}[$n$-stellige Relation]
		\label{def:nStelligeRelation}
		Basierend auf der vorangehenden Definition von \hyperref[def:Relation]{2-stelligen Relationen} definieren wir diese nun für $n$ Stellen: Seien $M_1$, ..., $M_n$ \hyperref[def:Menge]{Mengen} und $R \subseteq M_1 \times ... \times M_n$ dann heißt $R$ \textbf{$n$-stellige Relation} auf $M_1 \times ... \times M_n$.
	\end{definition}

	\section{Äquivalenzrelationen}

	\begin{definition}[Äquivalenzrelation]
		\label{def:Äquivalenzrelation}
		Unter einer \textbf{Äquivalenzrelation}\sidenote{wie bspw. $R_=$} verstehen wir eine \hyperref[def:Relation]{Relation} $R$ auf einer \hyperref[def:Menge]{Menge} $M$ welche folgende Eigenschaften erfüllt: 
		\begin{itemize}
			\item \textbf{Reflexivität}: Für alle Elemente $m$ aus $M$ gilt, dass diese mit sich selbst in Relation stehen: $$\forall m \in M: R(m,m)$$
			\item \textbf{Symmetrie}: Für alle Paare von Elementen $(m_1, m_2)$ aus $M \times M$ gilt, dass falls $m_1$ und $m_2$ in Relation stehen ($R(m_1,m_2)$) auch $m_2$ und $m_1$ in Relation stehen ($R(m_2, m_1)$): $$\forall m_1, m_2 \in M: R(m_1,m_2) \Leftrightarrow R(m_2, m_1)$$
			\item \textbf{Transitivität}: Für alle Elemente $m_1$, $m_2$ und $m_3$ aus $M$ gilt, dass falls $m_1$ und $m_2$ in Relation stehen ($R(m_1, m_2)$) und $m_2$ und $m_3$ in Relation stehen ($R(m_2,m_3)$) auch $m_1$ und $m_3$ in Relation stehen ($R(m_1, m_3)$): $$\forall m_1, m_2, m_3 \in M: (R(m_1,m_2) \land R(m_2,m_3)  ) \Rightarrow R(m_1,m_3) $$
		\end{itemize}
	\end{definition}

	\subsection{Äquivalenzklassen}
	
	\begin{definition}[Äquivalenzklasse]
		\label{def:Äquivalenzklasse}
		Sei $R$ eine \hyperref[def:Äquivalenzrelation]{Äquivalenzrelation} auf einer \hyperref[def:Menge]{Menge} $M$ und $m \in M$. Nun verstehen wir unter einer \textbf{Äquivalenzklasse} $[m]$ eine \hyperref[def:Menge]{Menge} von Elementen welche zu $m$ in Relation stehen (auch, \textit{die zu $m$ äquivalent sind}): $$[m]=\{n \in M | R(n,m)\}$$
	\end{definition}

%	\begin{definition}[kongruent modulo $n$, Restklasse]
%		\label{def:Restklasse}
%		Sei $R$ eine \hyperref[def:Äquivalenzrelation]{Äquivalenzrelation} auf $\mathbb{Z}$ bei der die Differenz zwischen den Elementen durch eine gegebene Zahl teilbar ist. Die beiden Elemente $z_1, z_2 \in \mathbb{Z}$ nennen wir \textbf{kongruent modulo} $n$ mit $n \in \mathbb{N}$ wenn $z_1 - z_2$ durch $n$ ohne Rest teilbar ist, geschrieben $z_1 \equiv z_2 \mod n$ \sidenote{Nicht zu verwechseln mit $x \mod y$ (gesprochen $x$ \textit{Modulo} $y$), dem verbleibenden Rest nach der Division von $x$ durch $y$}
%	\end{definition}

	\subsection{Einschub: Teilbarkeit}

	\begin{definition}[teilbar, Quotient]
		\label{def:teilbar}
		\label{def:Quotient}
		Es seien $z_1, z_2 \in \mathbb{Z}$ mit $z_2 \not = 0$. Wir sagen dass $z_1$ durch $z_2$ \textbf{teilbar} ist (geschrieben $z_2|z_1$, ''$z_2$ teilt $z_1$'') wenn ein \textbf{Quotient} $q \in \mathbb{Z}$ existiert sodass $z_1 = z_2 \cdot q$ gilt.
	\end{definition}

%	\begin{definition}[kongruent modulo $n$]
%		\label{def:kongruentModuloN}
%		Wir definieren eine neue \hyperref[def:Äquivalenzrelation]{Äquivalenzrelation}: Es seien $z_1,z_2 \in \mathbb{Z}$, $n \in \mathbb{N}^*$. Die beiden Elemente $z_1, z_2$ nennen wir \textbf{kongruent modulo} $n$ wenn $z_1 - z_2$ durch $n$ ohne Rest \hyperref[def:teilbar]{teilbar} ist, geschrieben $z_1 \equiv z_2 \mod n$. 
%	\end{definition}
	
	\begin{definition}[kongruent modulo $n$]
		\label{def:kongruentModuloN}
		Es seien $z_1,z_2 \in \mathbb{Z}$, $n \in \mathbb{N}^*$. Die beiden Elemente $z_1, z_2$ nennen wir \textbf{kongruent modulo} $n$ wenn $z_1 - z_2$ durch $n$ ohne Rest \hyperref[def:teilbar]{teilbar} ist, geschrieben $z_1 \equiv z_2 \mod n$. 
	\end{definition}

	\begin{theorem}
		\label{satz:kongruentModuloN}
		Seien $z_1, z_2 \in \mathbb{Z}$. Die beiden Elemente $z_1,z_2$ sind genau dann \hyperref[def:kongruentModuloN]{kongruent modulo} $n$ mit $n \in \mathbb{N}^*$ (also $z_1 \equiv z_2 \mod n$) wenn $z_1$ und $z_2$ nach der Division mit $n$ beide den gleichen Rest haben (also $z_1 \mod n = z_2 \mod n$).
		\begin{proof}
			Wir wollen zeigen: $$z_1 \equiv z_2 \mod n \Leftrightarrow z_1 \mod n = z_2 \mod n$$ Zeigen wir zunächst ''$\Leftarrow$'': Dazu schreiben wir $z_1$ und $z_2$ um: $z_1 = q_1 \cdot n + r$ und $z_2 = q_2 \cdot n + r$ mit den Quotienten $q_1,q_2$ und dem Rest $r$ (aus $z_1 \mod n = z_2 \mod n$ folgt dass beide den gleichen Rest $r$ haben müssen). Folglich lautet die Differenz $z_1-z_2$ der beiden Elemente: $z_1 - z_2 = q_1 \cdot n + r - (q_2 \cdot n + r) = q_1 \cdot n - q_2 \cdot n = (q_1 - q_2) \cdot n$. Diese Differenz ist durch $n$ teilbar, also folgt $z_1 \equiv z_2 \mod n$. \\ \\
			Nun zeigen wir den ''$\Rightarrow$'' Teil: Schreiben wir $z_1$ und $z_2$ erneut um: $z_1 = q_1 \cdot n + r_1$ und $z_2 = q_2 \cdot n + r_2$ mit den Quotienten $q_1,q_2$ und den Resten $r_1,r_2$ (aus $z_1 \equiv z_2 \mod n$ ist nicht direkt ersichtlich dass beide den gleichen Rest haben müssen). Des Weiteren wissen wir dass die Differenz $z_1-z_2$ ein Vielfaches von $n$ ist: $z_1-z_2 = q \cdot n$. Wir können die Differenz aber auch so schreiben: $z_1 - z_2 = q_1 \cdot n + r_1 - q_2 \cdot n + r_2 = (q_1-q_2) \cdot n + (r_1-r_2) (=q \cdot n)$. Da $(q_1-q_2) \cdot n$ offensichtlich ein Vielfaches von $n$ ist, muss auch $(r_1-r_2)$ ein Vielfaches von $n$ sein: $r_1-r_2=q_r \cdot n$. Durch Umformen erhalten wir: $r_1 = q_r\cdot n + r_2$. Da aber $r_1 < n$ sein muss (und der Rest $r_2$ positiv sein muss), muss $q_r=0$ gelten, woraus folgt: $r_1 = 0 \cdot n + r_2 = r_2$. Da wir die Reste durch Anwendung von Modulo erhalten, folgt: $z_1 \mod n = z_2 \mod n$.
		\end{proof}
	\end{theorem}

	\begin{theorem}
		Die Relation $\equiv$ ist eine Äquivalenzrelation.
		\begin{proof}\hspace*{1cm}
			\begin{itemize}
				\item Reflexivität: $z_1 \equiv z_1 \mod n$ ist offensichtlich wahr (ob $z_1-z_1$ oder $z_1-z_1$\sidenote{Nein, kein Tippfehler} durch $n$ \hyperref[def:teilbar]{teilbar} ist macht keinen Unterschied)
				\item Symmetrie: Aus $z_1 \equiv z_2 \mod n$ folgt $z_2 \equiv z_1 \mod n$, da dadurch schlichtweg das Vorzeichen der Differenz der Elemente (und folglich auch des Quotienten) umgedreht wird, der Rest aber unverändert bleibt. 
				\item Transitivität: Dass wenn $z_1 \equiv z_2 \mod n$ und $z_2 \equiv z_3 \mod n$ wahr sind auch $z_1 \equiv z_3 \mod n$ wahr ist, lässt sich dank Satz \ref{satz:kongruentModuloN} einfach zeigen: $z_1$ und $z_2$ haben offensichtlich nach der Division mit $n$ den gleichen Rest ($z_1 \mod n = z_2 \mod n$), gleiches gilt auch für $z_2$ und $z_3$. Wenn $(z_1 \mod n) = (z_2 \mod n)$ und $(z_2 \mod n) = (z_3 \mod n)$\sidenote{Klammern gesetzt um ''$=$'' als Äquivalenzrelation hervorzuheben}, dann muss auch $(z_1 \mod n) = (z_3 \mod n)$ gelten, woraus (dank Satz \ref{satz:kongruentModuloN}) folgt, dass $z_1 \equiv z_3 \mod n$ wahr ist.
			\end{itemize}
		\end{proof}
	\end{theorem}

	\subsection{Restklassen}
	\begin{definition}[Restklasse]
		Sei $n \in \mathbb{N}^*$ und $m \in \mathbb{N}$ mit $m < n$. Nun bilden jene Elemente welche zu $m$ kongruent modulo $n$ sind eine \hyperref[def:Äquivalenzklasse]{Äquivalenzklasse} $[m]_n$, eine sogenannte \textbf{Restklasse}. 
	\end{definition}
	

	\section{Ordnungsrelationen}

	\begin{definition}[partielle Ordnungsrelation]
		\label{def:partielleOrdnungsrelation}
		Unter einer \textbf{partiellen Ordnungsrelation}\sidenote{wie bspw. $R_\leq$} verstehen wir eine \hyperref[def:Relation]{Relation} $R$ auf einer \hyperref[def:Menge]{Menge} $M$ welche folgende Eigenschaften erfüllt:
		\begin{itemize}
			\item Reflexivität (Siehe \ref{def:Äquivalenzrelation})
			\item Transitivität (Siehe \ref{def:Äquivalenzrelation})
			\item \textbf{Anti-Symmetrie}: Für alle Paare von Elementen $(m_1, m_2)$ aus $M \times M$ gilt, dass falls $m_1$ und $m_2$ und auch $m_2$ und $m_1$ in Relation stehen ($R(m_1,m_2)$ und $R(m_2, m_1)$), $m_1$ und $m_2$ gleich sein müssen: $$\forall m_1, m_2 \in M: (R(m_1,m_2) \land R(m_2, m_1)) \Rightarrow m_1 = m_2$$
		\end{itemize}
	\end{definition}

	\begin{definition}[totale Ordnungsrelation]
		Unter einer \textbf{totalen Ordnungsrelation} verstehen wir eine \hyperref[def:Relation]{Relation} $R$ auf einer \hyperref[def:Menge]{Menge} $M$ welche neben den Eigenschaften der \hyperref[def:partielleOrdnungsrelation]{partiellen Ordnungsrelation} auch die folgende Eigenschaft erfüllt:
		\begin{itemize}
			\item \textbf{Totalität}: Für alle Paare von Elementen $m_1$, $m_2$ mit $m_1 \in M$ und $m_2 \in M$ gilt, dass entweder $m_1$ und $m_2$ in Relation stehen oder aber $m_2$ und $m_1$: $$\forall m_1, m_2 \in M: R(m_1,m_2) \lor R(m_2, m_1) $$
		\end{itemize}
	\end{definition}
	
\end{document}