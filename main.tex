\documentclass[a4paper,10pt,ngerman]{scrreprt}

% Deutsche Einstellungen
\usepackage[ngerman]{babel}

% Für UTF8
\usepackage[utf8]{inputenc}

% Schriftart für den Titel
\usepackage[T1]{fontenc}

% kaobook style font
\usepackage[p,theoremfont]{newpxtext}
% Funktioniert nicht (warum auch immer)
% \RequirePackage[vvarbb,smallerops,bigdelims]{newpxmath}

% Abstandsränder
\renewcommand*\chapterheadstartvskip{\vspace*{-\topskip}}
\usepackage{geometry}
\geometry{a4paper,left=20mm,right=20mm,top=20mm}

% Sidenotes
\usepackage{sidenotes}

% Schriftarten für Überschriften
\addtokomafont{chapter}{\rmfamily}
\addtokomafont{chapterentry}{\rmfamily}
\addtokomafont{section}{\rmfamily}
\addtokomafont{subsection}{\rmfamily}
\addtokomafont{subsubsection}{\rmfamily}

% Definition von Definitionen
\usepackage{amsthm}
\usepackage{hyperref}
\theoremstyle{definition}
\newtheorem{definition}{Definition}[section]

%\makeatletter
%\newenvironment{Definition}[1][\relax]
%{
%	\ifx\relax#1\definition\def\@currentlabel{Definition}
%	\else\definition[#1]\def\@currentlabel{\textit{#1}}
%	\fi
%}
%{\enddefinition}
%\makeatother

% Nach einem Absatz nicht einrücken
\setlength{\parindent}{0pt}

% Beginnen wir mit Kapitel 0 ;)
\setcounter{chapter}{-1}

% Metadaten
\title{Mathematik für 1nf0rmatiker:innen}
\author{Tobias Prisching}
\date{ }

% Für Tests...
\usepackage{blindtext}
% \usepackage{showframe}

% Sollte als letztes in der Preamble stehen
\usepackage{subfiles}

\begin{document}
	
	\maketitle
	
	\newpage

	\tableofcontents

	% Überschreiben der geometry settings da diese
	% nach dem Inhaltsverzeichnis in der Breite sidenotes
	% berücksichtigen sollen
	\newgeometry{
		a4paper,
		top=20mm,
		bottom=20mm,
		left=20mm,
		right=20mm,
		marginparsep=5mm,
		marginparwidth=30mm,
		includemp
	}
	

	\newpage
	\chapter*{Vorwort}\label{Vorwort}
	\addcontentsline{toc}{chapter}{Vorwort}	
	\subfile{Kapitel/Vorwort}
	
	\newpage
	\chapter{Allgemeines}\label{kapitel:Allgemeines}	
	\subfile{Kapitel/Allgemeines}	
	
	\newpage
	\chapter{Mengenlehre}\label{kapitel:Mengenlehre}	
	\subfile{Kapitel/Mengenlehre}
	
\end{document}