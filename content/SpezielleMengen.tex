\documentclass[../../main.tex]{subfiles}

\begin{document}
	
	\chapter{Spezielle Mengen}
	
		%\section{Natürliche Zahlen $\mathbb{N}$, Ganze Zahlen $\mathbb{Z}$ und Rationale Zahlen $\mathbb{Q}$}
		\section{Natürliche ($\mathbb{N}$), Ganze ($\mathbb{Z}$) und Rationale ($\mathbb{Q}$) Zahlen}
		
		\begin{definition}[Natürliche Zahlen]
			\label{def:NatürlicheZahlen}
			Wir definieren die Menge $\mathbb{N}$ der \textbf{natürlichen Zahlen} mithilfe des folgenden \hyperref[def:Axiomensystem]{Axiomensystems}, bekannt als die \textit{Peano-Axiome}\sidenote{nach Giuseppe Peano (1858-1932)}:
%			\begin{enumerate}
%				\item[(P1)] $1 \in \mathbb{N}$
%				\item[(P2)] Sei $n \in \mathbb{N}$, so hat $n$ genau einen Nachfolger $n'$ mit $n' \not=1$ und $n' \in \mathbb{N}$
%				\item[(P3)] Seien $n,m \in \mathbb{N}$ voneinander verschiedene natürliche Zahlen, so sind ihre Nachfolger $n'$ bzw. $m'$ ebenfalls voneinander verschieden ($n \not= m \Rightarrow n' \not=m'$)
%				\item[(P4)] Sei $M \subseteq \mathbb{N}$. Erfüllt $M$ die beiden Eigenschaften
%					\begin{itemize}
%						\item $1 \in M$
%						\item Sei $n \in M$, so ist der Nachfolger $n'$ von $n$ ebenfalls in $M$ ($n \in M \Rightarrow n' \in M$)
%					\end{itemize}
%				so gilt: $M = \mathbb{N}$
%			\end{enumerate}
			\begin{enumerate}
				\item Die Zahl 0 ist eine natürliche Zahl: $$0 \in \mathbb{N}$$
				\item Sei $n$ eine natürliche Zahl, so hat $n$ genau einen Nachfolger $n'$\sidenote{Unter dem Nachfolger $n'$ einer Zahl $n$ verstehen wir im Kontext dieser Definition $n+1$} welcher ebenfalls eine natürliche Zahl ist: $$\forall n \in \mathbb{N}: n' \in \mathbb{N}$$
				\item Sei $n$ eine natürliche Zahl, so hat $n$ genau einen Nachfolger $n'$ ungleich 0: $$\forall n \in \mathbb{N}: n' \not= 0$$
				\item Seien $n$ und $m$ natürliche Zahlen und $n'$ und $m'$ ihre respektiven Nachfolger, so gilt dass falls $n'$ und $m'$ gleich sind auch $n$ und $m$ gleich sind: $$\forall n,m \in \mathbb{N}: n' = m' \Rightarrow n=m$$
				\item Sei $M$ eine Menge. Enthält $M$ die Zahl 0 und für jede in $M$ enthaltene Zahl $n$ auch ihren Nachfolger $n'$, so ist die Menge der natürlichen Zahlen eine Teilmenge von $M$: $$\forall M: (0 \in M \land (n \in M) \Rightarrow (n' \in M)) \Rightarrow \mathbb{N} \subseteq M$$
			\end{enumerate}
			Es gibt verschiedene Formulierungen der Peano-Axiome denen man begegnet. Eine Weitere wäre beispielsweise:
			\begin{enumerate}
				\item $1 \in \mathbb{N}$
				\item Sei $n \in \mathbb{N}$, so hat $n$ genau einen Nachfolger $n'$ mit $n' \not=1$ und $n' \in \mathbb{N}$
				\item Seien $n,m \in \mathbb{N}$ voneinander verschiedene natürliche Zahlen, so sind ihre Nachfolger $n'$ bzw. $m'$ ebenfalls voneinander verschieden ($n \not= m \Rightarrow n' \not=m'$)
				\item Sei $M \subseteq \mathbb{N}$. Erfüllt $M$ die beiden Eigenschaften
					\begin{itemize}
						\item $1 \in M$
						\item Sei $n \in M$, so ist der Nachfolger $n'$ von $n$ ebenfalls in $M$ ($n \in M \Rightarrow n' \in M$)
					\end{itemize}
				so gilt: $M = \mathbb{N}$
			\end{enumerate}
			Wir sehen, dass diese Version der Axiome die Zahl 0 nicht zu den natürlichen Zahlen zählt. In weiterer Folge werden wir jedoch die Zahl 0 zu $\mathbb{N}$ hinzunehmen.\sidenote{Falls doch einmal notwendig werden wir $\mathbb{N}^*$ für $\mathbb{N}\setminus \{0\}$ verwenden.}
		\end{definition}
	
		\begin{definition}[Ganze Zahlen]
			\label{def:GanzeZahlen}
			Basierend auf der Menge der \hyperref[def:NatürlicheZahlen]{natürlichen Zahlen} definieren wir die Menge der \textbf{ganzen Zahlen } $\mathbb{Z}$: $$\mathbb{Z} = \{z|z \in \mathbb{N} \lor -z \in \mathbb{N}\}$$
		\end{definition}
	
		\begin{definition}[Rationale Zahlen]
			Basierend auf der Menge der \hyperref[def:NatürlicheZahlen]{natürlichen Zahlen} und der \hyperref[def:GanzeZahlen]{ganzen Zahlen} definieren wir die Menge der \textbf{rationalen Zahlen} $\mathbb{R}$: $$\mathbb{R} = \{r | r = \frac{z}{n}, z \in \mathbb{Z}, n \in \mathbb{N}^* \}$$
		\end{definition}
	
	

		
		
		\subsection*{Die Mächtigkeit von $\mathbb{N}$, $\mathbb{Z}$ und $\mathbb{Q}$}
		\begin{definition}[abzählbar unendlich]
			\label{def:abzählbarUnendlich}
			Intuitiv stelle wir fest dass es \textit{unendlich} viele natürliche Zahlen gibt, da es für jede beliebige natürliche Zahl $n$ einen Nachfolger $n+1$ gibt welcher ebenfalls eine natürliche Zahl ist und ebenfalls einen Nachfolger hat usw. Folglich hat die Menge $\mathbb{N}$ keine endliche Kardinalität, daher definieren wir $|\mathbb{N}|=\aleph_0$ (gesprochen: Aleph Null) und sagen, dass $\mathbb{N}$ \textbf{abzählbar unendlich} ist. 
		\end{definition}
		
		Nun zeigen wir, basierend auf den Definition \ref{def:gleichmächtig} und \ref{def:abzählbarUnendlich}, dass auch die Mengen $\mathbb{Z}$ und $\mathbb{Q}$ abzählbar unendlich sind. Dazu suchen wir uns \hyperref[def:Bijektiv]{bijektive} \hyperref[def:Abbildung]{Abbildungen}\sidenote{Siehe Kapitel \ref{chap:Abbildungen}} $f_Z: \mathbb{N} \rightarrow \mathbb{Z}$ und $f_Q: \mathbb{N} \rightarrow \mathbb{Q}$ um zu zeigen dass $\mathbb{N}$, $\mathbb{Z}$ und $\mathbb{Q}$ \hyperref[def:gleichmächtig]{gleichmächtig} sind:
		
		\begin{itemize}
			\item $f_Z: \mathbb{N} \rightarrow \mathbb{Z}$: Eine solche \hyperref[def:Funktion]{Funktion} wäre z.B.: $f_Z(0) = 0$, $f_Z(1) = 1$, $f_Z(2)=-1$, bei der wir $0$ auf $0$, die ungeraden Elemente von $\mathbb{N}$ auf die positiven Elemente von $\mathbb{Z}$, und alle weiteren geraden Elemente aus $\mathbb{N}$ (größer $0$ natürlich) auf die negativen Elemente von $\mathbb{Z}$ abbilden. Es folgt: $|\mathbb{Z}| = |\mathbb{N}| = \aleph_0$
			\item $f_Q: \mathbb{N} \rightarrow \mathbb{Q}$: Wir beginnen damit, Brüche systematisch in folgendem Schema aufzuschreiben und nacheinander über die eingezeichneten Diagonalen abzuzählen, wobei ungekürzte Brüche wie $\frac{2}{2}$ übersprungen werden:
			
			$$			
			\begin{matrix}
			\frac{1}{1} & \rightarrow & \frac{1}{2} &          & \frac{1}{3} & \rightarrow & \frac{1}{4} & ... \\
					    & \swarrow    &             & \nearrow &             & \swarrow    &             &     \\
			\frac{2}{1} &             & \frac{2}{2} &          & \frac{2}{3} &             & \frac{2}{4} & ... \\
			\downarrow  & \nearrow    &             & \swarrow &             & \nearrow    &             &     \\
			\frac{3}{1} &             & \frac{3}{2} &          & \frac{3}{3} &             & \frac{3}{4} & ... \\
			\vdots      &             & \vdots      &          & \vdots      &             & \vdots      & 
 			\end{matrix}
			$$
			
			Wir erhalten dadurch folgende \hyperref[def:Abbildung]{Abbildung}:
	
			$$			
			\begin{matrix}
			1 & 2           & 3 & 4           & 5 & 6 & 7 & 8 & \dots \\
			\downarrow & \downarrow & \downarrow & \downarrow & \downarrow & \downarrow & \downarrow & \downarrow & \downarrow & \\
			1 & \frac{1}{2} & 2 & 3 & \frac{1}{3} & \frac{1}{4} & \frac{2}{3} & \frac{3}{2} & \dots
			\end{matrix}
			$$			
			
			Des Weiteren bilden wir 0 auf 0 ab und fügen für jedes \hyperref[def:Bild]{Bild} zusätzlich dessen negatives Gegenstück hinzu, ähnlich wie bei der \hyperref[def:Funktion]{Funkition} $f_Z$: 
			
			$$			
			\begin{matrix}
			0 & 1 & 2           & 3 & 4           & 5 & 6 & 7 & 8 & 9 & 10 & \dots \\
			\downarrow & \downarrow & \downarrow & \downarrow & \downarrow & \downarrow & \downarrow & \downarrow & \downarrow & \downarrow & \downarrow & \downarrow & \\
			0 & 1 & -1 & \frac{1}{2} & -\frac{1}{2} & 2 & -2 & 3 & -3 & \frac{1}{3} & - \frac{1}{3} & \dots
			\end{matrix}
			$$
			
			Durch diese Vorschrift\sidenote{\raggedright Bekannt als \textit{Cantors erstes Diagonalargument}} erhalten wir die \hyperref[def:Bijektiv]{bijektive} \hyperref[def:Abbildung]{Abbildung} $f_Q$, aus welcher folgt: $|\mathbb{Q}| = |\mathbb{N}| = \aleph_0$
		
		\end{itemize}
		
		\section{Reelle Zahlen $\mathbb{R}$}
		Kommt später...
		$\mathfrak{c}$
		
		\begin{definition}[Reelle Zahlen]
			\label{def:ReelleZahlen}
		\end{definition}
	
		\begin{definition}[überabzählbar unendlich]
			\label{def:überabzählbarUnendlich}
			$\mathfrak{c}$
		\end{definition}
		
		
		
		\section{Komplexe Zahlen $\mathbb{C}$}
			\begin{definition}[imaginäre Einheit]
				\label{def:imaginäreEinheit}
				Unter der \textbf{imaginären Einheit} verstehen wir die Zahl $i$ und definieren diese als $i^2 = -1$. 
			\end{definition}
		
			\begin{definition}[Komplexe Zahlen]
				\label{def:KomplexeZahlen}
				Basierend auf der Menge der \hyperref[def:ReelleZahlen]{reellen Zahlen} und der \hyperref[def:imaginäreEinheit]{imaginären Einheit} definieren wir die Menge der \textbf{komplexen Zahlen} als $$\mathbb{C} = \{c | c = a + b \cdot i \text{ mit } a,b \in \mathbb{R} \}$$
			\end{definition}
		
			\begin{definition}[Realteil, Imaginärteil]
				\label{def:Realteil}
				\label{def:Imaginärteil}
				Sei $c \in \mathbb{C}$, also $c = a+b \cdot i$ mit $a,b \in \mathbb{R}$. Dann definieren wir den \textbf{Realteil} $\mathfrak{R}(c)$ der komplexen Zahl $c$ als $\mathfrak{R}(c) = a$ und den \textbf{Imaginärteil} $\mathfrak{I}(c)=b$\sidenote{$i$ gehört \textit{nicht} zum Imaginärteil:\\ $\mathfrak{I}(c)=b\not=b \cdot i$}
			\end{definition}
	
\end{document}