\documentclass[../../main.tex]{subfiles}

\begin{document}
	
	\chapter{Matrizen}
	
	\section{Allgemein}
	
	\begin{definition}[Matrix, Dimension einer Matrix, Spaltenindex, Zeilenindex]
		\label{def:Matrix}
		\label{def:DimensionMatrix}
		\label{def:Spaltenindex}
		\label{def:Zeilenindex}
		Sei $(K,+,\cdot)$ ein \hyperref[def:Körper]{Körper}. Die rechteckige, tabellen-artige Anordnung 
		
		$$M = \begin{pmatrix}
			k_{1,1} & \dots & k_{1,m} \\
			\vdots  & \ddots & \vdots \\
			k_{n,1} & \dots & k_{n,m}
		\end{pmatrix}$$
		
		$M$ von Elementen $k_{i,j}$ aus $K$ nennen wir eine \textbf{$n\times m$ Matrix} (plural: Matrizen), wobei wir $i$ als \textbf{Zeilenindex} und $j$ als \textbf{Spaltenindex} bezeichnen. Die Matrix $M$\sidenote{Es ist üblich für Matrizen große Buchstaben zu verwenden} besteht aus $m$ Spaltenvektoren\sidenote{Verweis ausstehend} bzw. $n$ Zeilenvektoren\sidenote{Verweis ausstehend}. Wir nennen $(n,m)$ die \textbf{Dimension $\textrm{dim}(M)$ der Matrix} $M$. Die \hyperref[def:Menge]{Menge} aller Matrizen (bestehend aus Elementen $k_{i,j} \in K$) mit Dimension $(n,m)$ schreiben wir als $K^{n \times m}$. 
	\end{definition}

	\begin{definition}[Matrixaddition]
		\label{def:Matrixaddition}
		Sei $(K,+,\cdot)$ ein \hyperref[def:Körper]{Körper} und $A,B\in K^{n\times m}$. Wir definieren die \textbf{Matrixaddition} $A+B$ als
		
		$$
		A+B = 
		\begin{pmatrix}
			a_{1,1} & \dots & a_{1,m} \\
			\vdots  & \ddots & \vdots \\
			a_{n,1} & \dots & a_{n,m}
		\end{pmatrix}
		+
		\begin{pmatrix}
			b_{1,1} & \dots & b_{1,m} \\
			\vdots  & \ddots & \vdots \\
			b_{n,1} & \dots & b_{n,m}
		\end{pmatrix}
		=
		\begin{pmatrix}
			a_{1,1}+b_{1,1} & \dots & a_{1,m}+b_{1,m} \\
			\vdots  & \ddots & \vdots \\
			a_{n,1}+b_{n,1} & \dots & a_{n,m}+b_{n,m}
		\end{pmatrix}
		$$
		
		\textit{Wichtig}: Die Addition zweier \hyperref[def:Matrix]{Matrizen} $A$ und $B$ ist nur möglich, wenn sie die gleichen \hyperref[def:DimensionMatrix]{Dimensionen} haben: $\textrm{dim}(A) = \textrm{dim}(B)$. Die Matrixaddition ist 
		\begin{itemize}
			\item \hyperref[def:kommutativ]{kommutativ}: $A+B = B+A$
			\item \hyperref[def:assoziativ]{assoziativ}: $(A+B)+C = A+(B+C)$
		\end{itemize}
	\end{definition}

	\begin{definition}[Matrixmultiplikation]
		\label{def:Matrixmultiplikation}
		Sei $(K,+,\cdot)$ ein \hyperref[def:Körper]{Körper} und seien $A\in K^{n \times m}$ und $B \in K^{m \times o}$ \hyperref[def:Matrix]{Matrizen}. Wir definieren die \textbf{Matrixmultiplikation}\sidenote{Merksatz für die Matrixmultiplikation: \textit{Zeile} (von $A$) \textit{mal Spalte} (von $B$)!} $A\times B$ als
		
		\begin{align*}
			A \times B
			&=
			\begin{pmatrix}
			a_{1,1} & \dots & a_{1,m} \\
			\vdots  & \ddots & \vdots \\
			a_{n,1} & \dots & a_{n,m}
			\end{pmatrix}
			\times
			\begin{pmatrix}
			b_{1,1} & \dots & b_{1,o} \\
			\vdots  & \ddots & \vdots \\
			b_{m,1} & \dots & b_{m,o}
			\end{pmatrix}
			= \\ 
			&= 
			\begin{pmatrix}
				a_{1,1} \cdot b_{1,1} + \dots + a_{1,m} \cdot b_{m,1} & \dots & a_{1,1} \cdot b_{1,o} + \dots + a_{1,m} \cdot b_{m,o} \\
				\vdots  & \ddots & \vdots \\
				a_{n,1} \cdot b_{1,1} + \dots + a_{n,m} \cdot b_{m,1} & \dots & a_{n,1} \cdot b_{1,o} + \dots + a_{n,m} \cdot b_{m,o}
			\end{pmatrix}
		\end{align*}
		
		\textit{Wichtig}: Die Multiplikation zweier \hyperref[def:Matrix]{Matrizen} $A$ und $B$ ist nur möglich, wenn die Anzahl von Spalten in $A$ mit der Anzahl von Zeilen in $B$ übereinstimmt. Daher ist die Matrixmultiplikation
		\begin{itemize}
			\item zwar \hyperref[def:assoziativ]{assoziativ}: $(A \times B) \times C = A \times (B \times C)$
			\item aber allgemein \textit{nicht} \hyperref[def:kommutativ]{kommutativ}: $A \times B \not = B \times A$
			\item \hyperref[def:distributiv]{distributiv} bezüglich der \hyperref[def:Matrixaddition]{Matrixaddition}: $A \times (B+C) = A \times B + A \times C$
		\end{itemize}
	
		Das Ergebnis der Matrixmultiplikation hat dann die \hyperref[def:DimensionMatrix]{Dimensionen} $$\textrm{dim}(A \times B) = (n,o)$$
		Daher ist $A \times B \in K^{n \times o}$. \\
		
	
		\textbf{Beispiel:}		
		$$
		\begin{pmatrix} 1 & 2 \\ 3 & 4 \\ 5 & 6 \\ 7 & 8\end{pmatrix} 
		\times 
		\begin{pmatrix} 9 & 10 & 11 \\ 12 & 13 & 14\end{pmatrix}
		=
		\begin{pmatrix} 
		1 \cdot 9 + 2 \cdot 12 & 1 \cdot 10 + 2 \cdot 13  & 1 \cdot 11 + 2 \cdot 14 \\ 
		3 \cdot 9 + 4 \cdot 12 & 3 \cdot 10 + 4 \cdot 13  & 3 \cdot 11 + 4 \cdot 14 \\
		5 \cdot 9 + 6 \cdot 12 & 5 \cdot 10 + 6 \cdot 13  & 5 \cdot 11 + 6 \cdot 14 \\
		7 \cdot 9 + 8 \cdot 12 & 7 \cdot 10 + 8 \cdot 13  & 7 \cdot 11 + 8 \cdot 14
		\end{pmatrix}
		$$
		
	\end{definition}
	
	\begin{definition}[Multiplikation einer Matrix mit einem Skalar]
		Sei $(K,+,\cdot)$ ein \hyperref[def:Körper]{Körper} und sei $A\in K^{n \times m}$ eine beliebige \hyperref[def:Matrix]{Matrix} und $b \in K$ ein Skalar. Wir definieren die \textbf{Multiplikation einer Matrix $A$ mit einem Skalar $b$} als 
		
		$$A \cdot b = b \cdot A = b \cdot \begin{pmatrix} a_{1,1} & \dots & a_{1,m} \\ \vdots & \ddots & \vdots \\ a_{n,1} & \dots & a_{n,m} \end{pmatrix} = \begin{pmatrix} b \cdot a_{1,1} & \dots & b \cdot a_{1,m} \\ \vdots & \ddots & \vdots \\ b \cdot a_{n,1} & \dots & b \cdot a_{n,m} \end{pmatrix}$$
		
		Dabei wird jedes Element $a_{i,j}$ der \hyperref[def:Matrix]{Matrix} $A$ mit dem Skalar $b$ multipliziert. Andersherum kann so jeder beliebige Skalar (ungleich 0) aus $A$ herausgehoben werden. 
	\end{definition}

	\begin{definition}[Transponieren, transponierte Matrix]
		\label{def:Transponieren}
		\label{def:transponierteMatrix}
		Sei $(K,+,\cdot)$ ein \hyperref[def:Körper]{Körper} und sei $A\in K^{n \times m}$ eine beliebige \hyperref[def:Matrix]{Matrix}, so verstehen wir unter der \textbf{transponierten Matrix} $A^\top \in K^{m \times n}$ jene \hyperref[def:Matrix]{Matrix} welche durch Vertauschung der Zeilen und Spalten von $A$ entsteht.
		
		$$A = \begin{pmatrix} a_{1,1} & \dots & a_{1,m} \\ \vdots & \ddots & \vdots \\ a_{n,1} & \dots & a_{n,m} \end{pmatrix} \Rightarrow A^\top = \begin{pmatrix} a_{1,1} & \dots & a_{1,n} \\ \vdots & \ddots & \vdots \\ a_{m,1} & \dots & a_{n,m} \end{pmatrix}$$
		
		Für eine \hyperref[def:Matrix]{Matrix} $A \in K^{n \times n}$ entspricht dies der Spiegelung entlang der \textit{Hauptdiagonale}. 
		
		\textbf{Beispiel:}
		
		$$A = \begin{pmatrix} 1 & 2 \\ 3 & 4 \\ 5 & 6 \end{pmatrix} \Rightarrow A^\top = \begin{pmatrix} 1 & 3 & 5 \\ 2 & 4 & 6 \end{pmatrix}$$
	\end{definition}

	\begin{theorem}
		Sei $(K,+,\cdot)$ ein \hyperref[def:Körper]{Körper} und sei $A\in K^{n \times m}$ eine beliebige \hyperref[def:Matrix]{Matrix}. Es gilt: $$(A^\top)^\top = A$$
	\end{theorem}

	\begin{theorem}
		Sei $(K,+,\cdot)$ ein \hyperref[def:Körper]{Körper} und seien $A_1,...,A_n$ beliebige \hyperref[def:Matrix]{Matrizen} passender Größe über $K$. Es gilt: $$(A_1 \times ... \times A_n)^\top = A_n^\top \times ... \times A_1^\top$$
	\end{theorem}

	\section{Spezielle Matrizen und ihre Eigenschaften}
	
	\begin{definition}[Quadratische Matrix]
		\label{def:QuadratischeMatrix}
		Sei $(K,+,\cdot)$ ein \hyperref[def:Körper]{Körper}. Wir nennen \hyperref[def:Matrix]{Matrizen} $A \in K^{n \times n}$ \textbf{quadratisch} da diese gleich viele Zeilen wie Spalten haben. 
	\end{definition}
	
	
	\begin{definition}[Symmetrische Matrix]
		\label{def:SymmetrischeMatrix}
		Sei $(K,+,\cdot)$ ein \hyperref[def:Körper]{Körper} und sei $A\in K^{n \times n}$ eine beliebige \hyperref[def:QuadratischeMatrix]{quadratische Matrix}. Wir nennen $A$ eine \textbf{symmetrische Matrix} wenn gilt: $A=A^\top$. 
	\end{definition}

	\begin{definition}[Hermitesche Matrix]
		\label{def:HermitescheMatrix}
		Sei $A \in \mathbb{C}^{n \times n}$ eine \hyperref[def:QuadratischeMatrix]{quadratische Matrix} mit Elementen aus $\mathbb{C}$. Wir nennen $A$ eine \textbf{hermitesche Matrix} wenn gilt: $$A^\top = \overline{A} \Rightarrow \overline{A}^\top = A$$ wobei $\overline{A}$ die \hyperref[def:komplexKonjugiert]{komplex konjugierten} Versionen $\overline{a_{i,j}}$ der Einträge $a_{i,j}$ von $A$ enthält. \textit{Wichtig}: Die Einträge entlang der \textit{Hauptdiagonale} von $A$ müssen \hyperref[def:ReelleZahlen]{reell} sein. 
	\end{definition}

	\begin{theorem}
		Seien $A_1,...,A_n$ beliebige \hyperref[def:HermitescheMatrix]{hermitesche Matrizen} passender Größe über $\mathbb{C}$. Es gilt: $$(\overline{A_1 \times ... \times A_n})^\top = \overline{A_n}^\top \times ... \times \overline{A_1}^\top$$
	\end{theorem}

	\begin{definition}[Rang einer Matrix]
		\label{def:RangMatrix}
		Sei $(K,+,\cdot)$ ein \hyperref[def:Körper]{Körper} und sei $A\in K^{n \times m}$ eine beliebige \hyperref[def:Matrix]{Matrix}. Als \textbf{Rang der Matrix} \textrm{rang}(A) bezeichnen wir die maximale Anzahl linear unabhängiger Zeilen- bzw. Spaltenvektoren.
	\end{definition}

	\begin{theorem}
		Sei $(K,+,\cdot)$ ein \hyperref[def:Körper]{Körper} und sei $A\in K^{n \times m}$ eine beliebige \hyperref[def:Matrix]{Matrix}. Es gilt: Der ''Zeilenrang'' von $A$ ist \textit{immer} gleich dem ''Spaltenrang'' von $A$. Es folgt: $$\textrm{rang}(A) = \textrm{rang}(A^\top)$$
	\end{theorem}

	\begin{theorem}
		Sei $(K,+,\cdot)$ ein \hyperref[def:Körper]{Körper} und sei $A\in K^{n \times m}$ eine beliebige \hyperref[def:Matrix]{Matrix}. Es gilt: Der \hyperref[def:RangMatrix]{Rang} von $A$ kann maximal dem Minimum von Zeilen- und Spaltenanzahl sein: $$\textrm{rang}(A) \leq \textrm{min}(n,m)$$
	\end{theorem}

	\begin{definition}[Reguläre/Singuläre Matrix, voller Rang]
		\label{def:reguläreMatrix}
		\label{def:singuläreMatrix}
		\label{def:vollerRang}
		Sei $(K,+,\cdot)$ ein \hyperref[def:Körper]{Körper} und sei $A\in K^{n \times n}$ eine beliebige \hyperref[def:QuadratischeMatrix]{quadratische Matrix}. Ist der \hyperref[def:RangMatrix]{Rang} von $A$ gleich ihrer \hyperref[def:DimensionMatrix]{Zeilen- bzw. Spaltendimension} ($\textrm{rang}(A)=n$), so sagen wir $A$ hat \textbf{vollen Rang} und bezeichnen $A$ als eine \textbf{reguläre Matrix}. \hyperref[def:Matrix]{Matrizen} welche nicht vollen Rang haben, nennen wir \textbf{singulär}. 
	\end{definition}
	
\end{document}