\documentclass[../../main.tex]{subfiles}

\begin{document}
	
	\chapter{Relationen}
	
	\begin{definition}[(2-stellige) Relation]
		\label{def:Relation}
		Seien $M$ und $N$ \hyperref[def:Menge]{Mengen} und $R$ eine \hyperref[def:Teilmenge]{Teilmenge} des kartesischen Produkt der beiden ($R \subseteq M \times N$). Nun verstehen wir unter $R$ eine \textbf{Relation} auf $M \times N$. Für den Spezialfall $M=N$ heißt $R$ Relation auf $M$.\sidenote{Der Punkt dabei ist, dass die \hyperref[def:Teilmenge]{Teilmenge} $R$ beliebig definiert werden kann um verschiedenste Relationen bilden zu können.}
	\end{definition}

	Dabei gilt es vor allem zu beachten, dass eine Relation für ein Paar von Werten (= ein Element des kartesischen Produkts) nur entweder zutreffen kann oder nicht - entweder es gibt eine Relation zwischen den Werten oder nicht. Wir können eine Relation also als eine Aussagevorschrift über das Verhältnis zwischen diesen Werten betrachten.\\
	
	
	Um auszudrücken dass beispielsweise die Elemente $0$ und $1$ die Relation $<$ erfüllen können wir eine der folgenden Schreibweisen verwenden: $R_<(0,1)$\sidenote{Es kann auch nur $R(0,1)$ geschrieben werden, vorausgesetzt die Bezeichnung $R$ ist eindeutig} oder $(0,1) \in R_<$ oder schlicht $0<1$.
	
	\begin{definition}[$n$-stellige Relation]
		\label{def:nStelligeRelation}
		Basierend auf der vorangehenden Definition von \hyperref[def:Relation]{2-stelligen Relationen} definieren wir diese nun für $n$ Stellen: Seien $M_1$, ..., $M_n$ \hyperref[def:Menge]{Mengen} und $R \subseteq M_1 \times ... \times M_n$ dann heißt $R$ \textbf{$n$-stellige Relation} auf $M_1 \times ... \times M_n$.
	\end{definition}

	\begin{definition}[Äquivalenzrelation]
		\label{def:Äquivalenzrelation}
		Unter einer \textbf{Äquivalenzrelation}\sidenote{wie bspw. $R_=$} verstehen wir eine \hyperref[def:Relation]{Relation} $R$ auf einer \hyperref[def:Menge]{Menge} $M$ welche folgende Eigenschaften erfüllt: 
		\begin{itemize}
			\item \textbf{Reflexivität}: Für alle Elemente $m$ aus $M$ gilt, dass diese mit sich selbst in Relation stehen: $$\forall m \in M: R(m,m)$$
			\item \textbf{Symmetrie}: Für alle Paare von Elementen $(m_1, m_2)$ aus $M \times M$ gilt, dass falls $m_1$ und $m_2$ in Relation stehen ($R(m_1,m_2)$) auch $m_2$ und $m_1$ in Relation stehen ($R(m_2, m_1)$): $$\forall m_1, m_2 \in M: R(m_1,m_2) \Leftrightarrow R(m_2, m_1)$$
			\item \textbf{Transitivität}: Für alle Elemente $m_1$, $m_2$ und $m_3$ aus $M$ gilt, dass falls $m_1$ und $m_2$ in Relation stehen ($R(m_1, m_2)$) und $m_2$ und $m_3$ in Relation stehen ($R(m_2,m_3)$) auch $m_1$ und $m_3$ in Relation stehen ($R(m_1, m_3)$): $$\forall m_1, m_2, m_3 \in M: (R(m_1,m_2) \land R(m_2,m_3)  ) \Rightarrow R(m_1,m_3) $$
		\end{itemize}
	\end{definition}

	\begin{definition}[Äquivalenzklasse]
		Sei $R$ eine \hyperref[def:Äquivalenzrelation]{Äquivalenzrelation} auf einer \hyperref[def:Menge]{Menge} $M$ und $m \in M$. Nun verstehen wir unter einer \textbf{Äquivalenzklasse} $[m]$ eine \hyperref[def:Menge]{Menge} von Elementen welche zu $m$ in Relation stehen (auch, \textit{die zu $m$ äquivalent sind}): $$[m]=\{n \in M | R(n,m)\}$$
	\end{definition}

	\begin{definition}[partielle Ordnungsrelation]
		\label{def:partielleOrdnungsrelation}
		Unter einer \textbf{partiellen Ordnungsrelation}\sidenote{wie bspw. $R_\leq$} verstehen wir eine \hyperref[def:Relation]{Relation} $R$ auf einer \hyperref[def:Menge]{Menge} $M$ welche folgende Eigenschaften erfüllt:
		\begin{itemize}
			\item Reflexivität (Siehe \ref{def:Äquivalenzrelation})
			\item Transitivität (Siehe \ref{def:Äquivalenzrelation})
			\item \textbf{Anti-Symmetrie}: Für alle Paare von Elementen $(m_1, m_2)$ aus $M \times M$ gilt, dass falls $m_1$ und $m_2$ und auch $m_2$ und $m_1$ in Relation stehen ($R(m_1,m_2)$ und $R(m_2, m_1)$), $m_1$ und $m_2$ gleich sein müssen: $$\forall m_1, m_2 \in M: (R(m_1,m_2) \land R(m_2, m_1)) \Rightarrow m_1 = m_2$$
		\end{itemize}
	\end{definition}

	\begin{definition}[totale Ordnungsrelation]
		Unter einer \textbf{totalen Ordnungsrelation} verstehen wir eine \hyperref[def:Relation]{Relation} $R$ auf einer \hyperref[def:Menge]{Menge} $M$ welche neben den Eigenschaften der \hyperref[def:partielleOrdnungsrelation]{partiellen Ordnungsrelation} auch die folgende Eigenschaft erfüllt:
		\begin{itemize}
			\item \textbf{Totalität}: Für alle Paare von Elementen $m_1$, $m_2$ mit $m_1 \in M$ und $m_2 \in M$ gilt, dass entweder $m_1$ und $m_2$ in Relation stehen oder aber $m_2$ und $m_1$: $$\forall m_1, m_2 \in M: R(m_1,m_2) \lor R(m_2, m_1) $$
		\end{itemize}
	\end{definition}
	
\end{document}