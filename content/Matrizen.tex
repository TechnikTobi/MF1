\documentclass[../../main.tex]{subfiles}

\begin{document}
	
	\chapter{Matrizen}
	
	\begin{definition}[Matrix, Dimension einer Matrix, Spaltenindex, Zeilenindex]
		\label{def:Matrix}
		\label{def:DimensionMatrix}
		\label{def:Spaltenindex}
		\label{def:Zeilenindex}
		Sei $(K,+,\cdot)$ ein \hyperref[def:Körper]{Körper}. Die rechteckige, tabellen-artige Anordnung 
		
		$$m = \begin{pmatrix}
			k_{1,1} & \dots & k_{1,m} \\
			\vdots  & \ddots & \vdots \\
			k_{n,1} & \dots & k_{n,m}
		\end{pmatrix}$$
		
		$m$ von Elementen $k_{i,j}$ aus $K$ nennen wir eine \textbf{$n\times m$ Matrix} (plural: Matrizen), wobei wir $i$ als \textbf{Zeilenindex} und $j$ als \textbf{Spaltenindex} bezeichnen. Sie besteht aus $m$ Spaltenvektoren\sidenote{Verweis ausstehend} bzw. $n$ Zeilenvektoren\sidenote{Verweis ausstehend}. Wir nennen $(n,m)$ die \textbf{Dimension der Matrix} $m$. Die \hyperref[def:Menge]{Menge} aller Matrizen (bestehend aus Elementen $k_{i,j} \in K$) mit Dimension $(n,m)$ schreiben wir als $K^{n \times m}$. 
	\end{definition}
	
\end{document}