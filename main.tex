\documentclass[a4paper,10pt,ngerman]{scrreprt}

% Deutsche Einstellungen
\usepackage[ngerman]{babel}

% Für UTF8
\usepackage[utf8]{inputenc}

% Schriftart für den Titel
\usepackage[T1]{fontenc}

% kaobook style font
\usepackage[p,theoremfont]{newpxtext}
% Funktioniert nicht (warum auch immer)
% \RequirePackage[vvarbb,smallerops,bigdelims]{newpxmath}

% Abstandsränder
\renewcommand*\chapterheadstartvskip{\vspace*{-\topskip}}
\usepackage{geometry}
\geometry{
	a4paper,
	top=20mm,
	bottom=20mm,
	left=20mm,
	right=20mm,
	marginparsep=5mm,
	marginparwidth=37mm,
	includemp
}

% Sidenotes
\usepackage{sidenotes}

% Für mathematische Notationen
\usepackage{amssymb}
\usepackage{amsmath}

\usepackage{longtable, array, booktabs}

% Schriftarten für Überschriften
\addtokomafont{title}{\rmfamily}
\addtokomafont{part}{\rmfamily}
\addtokomafont{chapter}{\rmfamily}
\addtokomafont{chapterentry}{\rmfamily}
\addtokomafont{section}{\rmfamily}
\addtokomafont{subsection}{\rmfamily}
\addtokomafont{subsubsection}{\rmfamily}

% Definition von Definitionen
\usepackage{amsthm}
\usepackage{hyperref}
\theoremstyle{definition}
\newtheorem{definition}{Definition}[section]
\newtheorem{theorem}{Satz}[section]

% Für große Matrizen
\setcounter{MaxMatrixCols}{20}

% Nach einem Absatz nicht einrücken
\setlength{\parindent}{0pt}

% Metadaten
\title{Mathematik für 1nf0rmatiker:innen}
\author{Tobias Prisching}
\date{Fassung vom \today}

% Beginnen wir mit Kapitel 0 ;)
% \setcounter{chapter}{-1}
\setcounter{chapter}{0}

% Für Tests...
\usepackage{blindtext}
% \usepackage{showframe}

% Sollte als letztes in der Preamble stehen
\usepackage{subfiles}

\begin{document}
	
	\newgeometry{a4paper,left=20mm,right=20mm,top=20mm}
	
	\maketitle
	
	\newpage

	\tableofcontents
	
	% Alte geometry wiederherstellen
	\clearpage
	\restoregeometry

	\newpage
	\chapter*{Vorwort}\label{Vorwort}
	\addcontentsline{toc}{chapter}{Vorwort}	
	\subfile{content/Vorwort}
	
	\newpage
	\chapter*{Symbole}\label{Symbole}
	\addcontentsline{toc}{chapter}{Symbole}	
	\subfile{content/Symbole}
	
	\newpage
	\addpart{Logik}
	\subfile{content/GrundlagenDerLogik}	
	\subfile{content/Beweistechniken}
	
	\newpage
	\addpart{Mengen und Relationen}
	\subfile{content/Mengenlehre}
	\subfile{content/SpezielleMengen}
	\subfile{content/Relationen}
	
	\newpage
	\addpart{Funktionale Abhängigkeiten}
	\subfile{content/Abbildungen}
	
	\newpage
	\addpart{Algebra}
	\subfile{content/AlgebraischeStrukturen}
	\subfile{content/Polynome}
	\subfile{content/Vektorraeume}
	
\end{document}