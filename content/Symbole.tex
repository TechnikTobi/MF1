\documentclass[../main.tex]{subfiles}

\begin{document}
	
	\newcommand{\newNotationRow}{\\[2.5mm]}
	
	\begin{longtable}{p{30mm}p{60mm}p{65mm}} %longtable für mehrseitige Tabelle, sonst tabular
		\toprule 
		
		\centering Symbol & Bedeutung & Beispiel \newNotationRow
		
		\midrule
				
		%\centering $\forall$ & für alle & $\forall x \in \mathbb{N}^*: x > 0$ \newNotationRow
		
		\centering $w$, $\top$ & logisches wahr (Tautologie) & - \newNotationRow
		
		\centering $f$, $\bot$ & logisches falsch (Antilogie) & - \newNotationRow
				
		\centering $\lnot$ & logische Negation & $\lnot A$ \newNotationRow
		
		\centering $\land$ & logische Konjunktion (Und/AND) & $A \land B$ \newNotationRow
		
		\centering $\lor$ & logische Disjunktion (Oder/OR) & $\textrm{ToBe} \lor \lnot \textrm{ToBe}$ \newNotationRow
		
		\centering $\tilde\land$ & logisches Nicht-Und (NAND) & $A \tilde\land B$ \newNotationRow
		
		\centering $\tilde\lor$ & logisches Nicht-Oder (NOR) & $A \tilde\lor B$ \newNotationRow
		
		\centering $\veebar$, $\dot{\lor}$ & logisches exklusives Oder (XOR) & $A \veebar B$ \newNotationRow
		
		\centering $\Rightarrow$ & logische Implikation & $A \Rightarrow B$ \newNotationRow
		
		\centering $\Leftrightarrow$ & logische Äquivalenz & $A \Leftrightarrow B$ \newNotationRow
		
		\bottomrule
		
		\caption{Logik Symbole}
	\end{longtable}









	\begin{longtable}{p{30mm}p{60mm}p{65mm}} %longtable für mehrseitige Tabelle, sonst tabular
		\toprule 
		
		\centering Symbol & Bedeutung & Beispiel \newNotationRow
		
		\midrule
		
		\centering $\in$ & ist Element von & $x \in M$ \newNotationRow
		
		\centering $\not\in$ & ist nicht Element von & $y \not\in M$ \newNotationRow
		
		\centering $\subseteq$ & ist Teilmenge von & $N \subseteq M$ \newNotationRow
		
		\centering $\subset, \subsetneq, \subsetneqq$ & ist echte Teilmenge von & $N \subset M$ \newNotationRow
		
		\centering $\not\subseteq$ & ist nicht Teilmenge von & $N \not\subseteq M$ \newNotationRow
		
		\centering $\supseteq$ & ist Obermenge von & $M \supseteq N$ \newNotationRow
		
		\centering $\supset, \supsetneq, \supsetneqq$ & ist echte Obermenge von & $M \supset N$ \newNotationRow
		
		\centering $\not\supseteq$ & ist nicht Obermenge von & $M \not\supseteq N$ \newNotationRow
		
		\centering $\mathcal{P}$ & Potenzmenge & $\mathcal{P}(\{0,1\}) = \{\emptyset, \{0\}, \{1\}, \{0,1\}\}$ \newNotationRow
		
		\centering $\cap$ & Durchschnitt & $M \cap N$ \newNotationRow
		
		\centering $\cup$ & Vereinigung & $M \cup N$ \newNotationRow
		
		\centering $\setminus$ & Differenz & $M \setminus N$ \newNotationRow
		
		\centering $\overline{M}, M^\mathrm{C}$ & Komplement & $M^\mathrm{C} = \overline{M}$ \newNotationRow
				
		\bottomrule
		
		\caption{Mengen Symbole}
	\end{longtable}

\end{document}

